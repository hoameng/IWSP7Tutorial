%%%
% IEEGPortal Tutorial - IWSP7 
% 7/17/2015
%
%%%
\documentclass[10pt]{article}

\usepackage[margin=1in]{geometry}
\usepackage{enumitem}
\usepackage{hyperref}

\begin{document}

\title{International Epilepsy Electrophysiology (IEEG) Portal \\
\url{www.ieeg.org} \\
	\author { \large IWSP7 Tutorial \\
	August 3rd, 2015} }
\date{}
\maketitle

\section*{Description}
The IEEG Portal is a cloud-based web application that allows you to easily view and access data in real-time, through a web browser or through programming environments such as Matlab. \\
The first part of the tutorial will cover an introduction as well as a hands-on overview of the web and Matlab interfaces. Following the intermission, we will have a mini seizure detection project that you can work on individually or in groups to highlight the strengths of the portal.

\section*{Preparation}
Completing the steps below will allow us to get started quickly without leaving anyone behind.
\begin{enumerate}
\item Register for an account according to directions below
\item Ensure Matlab is working on your computer (temporary Matlab Online licenses will be provided by the University of Melbourne for those who need it)
\item Download the Matlab and CLI toolboxes \url{https://code.google.com/p/braintrust/wiki/Downloads}
\item Download Utilities from GitHub \url{https://github.com/ieeg-portal/portal-matlab-tools}
\item Bring a dataset from your lab for assistance with converting and uploading to the database for public or private use.  (optional)
\item Bring a seizure detector for the second part of the tutorial.  (optional) 
\end{enumerate}

\section*{Account Creation}
In order to gain access to the datasets you must create an account on the IEEG Portal. You will be granted access to interact with real clinical and research data obtained from humans and a variety of animal models.

\begin{enumerate}
    \item Go to \url{http://main.ieeg.org}.
    \item Click "Create New Account"
    \item Fill out required fields. {\bf For {\bf Data Use Purposes}, enter {\bf IWSP7} on the first line}
    \item You will receive an email once your account is ready to access datasets (this validation is not immediate).
\end{enumerate}

\end{document}
